% !TEX TS-program = xelatex
% !TEX encoding = UTF-8 Unicode
% !Mode:: "TeX:UTF-8"

\documentclass{resume}
\usepackage{zh_CN-Adobefonts_external} % Simplified Chinese Support using external fonts (./fonts/zh_CN-Adobe/)
% \usepackage{NotoSansSC_external}
% \usepackage{NotoSerifCJKsc_external}
% \usepackage{zh_CN-Adobefonts_internal} % Simplified Chinese Support using system fonts
\usepackage{linespacing_fix} % disable extra space before next section
\usepackage{cite}

\begin{document}
\pagenumbering{gobble} % suppress displaying page number

\name{黎静北}

\basicInfo{
  \email{i@jingbei.li} \textperiodcentered\ 
  \phone{(+86) 13389921221} \textperiodcentered\ 
  \linkedin[jingbei-li]{https://www.linkedin.com/in/jingbei-li}
}

\section{\faGraduationCap\  教育背景}
\datedsubsection{\textbf{清华大学}, 北京}{2015 -- 2018}
\textit{工学硕士}\ 计算机科学与技术系,计算机科学与技术
\datedsubsection{\textbf{清华大学}, 北京}{2011 -- 2015}
\textit{工学学士}\ 计算机科学与技术系,计算机科学与技术

\section{\faUsers\ 项目、科研经历}
\datedsubsection{\textbf{基于多模态多尺度特征的口语表现力评测}}{2015年9月 -- 2018年5月}
\role{硕士学位论文}{指导教师:吴志勇}
\begin{itemize}
	\item Li J, Wu Z, Li R, Xu M, Lei K, Cai L, Multi-modal Multi-scale Speech Expression Evaluation in Computer-Assisted Language Learning [C] //Proceedings of 2018 International Conference on AI and Mobile Services (AIMS 2018), 2018.
	\item 提出了多模态稀疏自编码器 (Multi-modal Sparse Auto Encoder, MSAE)
	\item 提出了一种基于循环自编码器 (Recurrent Auto Encoder, RAE) 的低失真率的池化降采样方法
\end{itemize}

\datedsubsection{\textbf{中英文空乘广播口语能力综合评测}}{2015年9月 -- 2018年5月}
\role{硕士期间参与项目}{指导教师:吴志勇、徐明星}
\begin{itemize}
	\item 实现了基于全局特征提取和支持向量回归(Support Vector Regression, SVR)的口语能力综合评测
	\item 实现了基于高斯混合模型(Gaussian Mixture Model, GMM)和统一背景模型(Universal Background Model, UBM)的口语能力综合评测
	\item 实现了基于多模态多尺度特征的口语表现力评测
\end{itemize}

\datedsubsection{\textbf{搜狗语音助手情感评测项目}}{2017年9月 -- 2018年1月}
\role{硕士期间参与项目}{指导教师:吴志勇、贾珈}
\begin{itemize}
	\item Li R, Wu Z, Jia J, et al. Inferring User Emotive State Changes in Realistic Human-Computer Conversational Dialogs[C]//2018 ACM Multimedia Conference on Multimedia Conference. ACM, 2018: 136-144.
	\item 第四作者,负责实现了绝大部分实验
\end{itemize}

\datedsubsection{\textbf{中科院自动化研究所模式识别重点实验室}}{2014年9月 -- 2015年6月}
\role{本科期间参与项目}{指导教师:陶建华、李雅}
\begin{itemize}
  \item 语音合成方向
  \item 提出了一种通过二次曲线对基频曲线进行建模的方式
  \item 提出了一种通过低通滤波得到语势曲线的方式
\end{itemize}

\datedsubsection{\textbf{社交媒体中的文字与图片情感分析}}{2014年1月 -- 2015年8月}
\role{学士学位论文}{指导教师:张敏}
\begin{itemize}
  \item 大学三年级时完成
\end{itemize}

\datedsubsection{\textbf{基于网络算法描述影响力与被影响力}}{2014年2月 -- 2014年2月}
\role{比赛项目}{2014年美国数学建模大赛参赛项目}
\begin{itemize}
	\item 获2014年美国数学建模大赛参赛二等奖
\end{itemize}

\datedsubsection{\textbf{基于树莓派开发的简易交互机器人}}{2015年11月 -- 2016年1月}
\role{课程大作业}{https://jingbei.li/2016/01/03/raspberry-pi-android-robot}
\begin{itemize}
	\item 可以识别人脸并追踪旋转头部
	\item 可以识别简单的语音命令并回答
\end{itemize}

\datedsubsection{\textbf{基于Pebble智能手表的简易英德语音翻译软件}}{2015年12月 -- 2016年1月}
\role{课程大作业}{https://github.com/petronny/pebble-translator}

\datedsubsection{\textbf{平板电脑用蓝牙背板键盘Geek's Keyboard}}{2014年11月 -- 2014年11月}
\role{比赛项目}{THACKS Hackathon参赛项目}
\begin{itemize}
	\item 获THACKS Hackathon第一名
\end{itemize}

\datedsubsection{\textbf{Android版泡泡堂游戏}}{2013年9月 -- 2013年10月}
\role{课程大作业}{https://github.com/petronny/SEBomber}
\begin{itemize}
	\item 基于cocos2d-x开发
\end{itemize}

\datedsubsection{\textbf{Archlinux中文社区软件源archlinuxcn}}{2014年10月 -- 至今}
\role{维护者}{https://github.com/archlinuxcn}
\begin{itemize}
	\item 现维护约60/1597个软件包
	\item 参与改进打包机器人lilac (https://github.com/archlinuxcn/lilac)
\end{itemize}

\datedsubsection{\textbf{Archlinux科研软件源arch4edu}}{2016年8月 -- 至今}
\role{维护者}{https://github.com/arch4edu}
\begin{itemize}
	\item 现维护约316/322个软件包
	\item 同时提供对ArchlinuxARM的支持
\end{itemize}

% Reference Test
%\datedsubsection{\textbf{Paper Title\cite{zaharia2012resilient}}}{May. 2015}
%An xxx optimized for xxx\cite{verma2015large}
%\begin{itemize}
%  \item main contribution
%\end{itemize}

\section{\faCogs\ IT 技能}
\begin{itemize}[parsep=0.5ex]
	\item 编程语言: Python, C, C++, Shell > Java, MATLAB >  JavaScript, Perl, PHP
	\item 开发平台: Linux, Android, 树莓派
	\item 算法知识: 基本算法知识和数据结构,并行编程,常见机器学习算法
\end{itemize}

%\section{\faHeartO\ 获奖情况}
%\datedline{\textit{第一名}, THACKS Hackathon}{2014 年 11月}
%\datedline{其他奖项}{2015}

\section{\faInfo\ 其他}
% increase linespacing [parsep=0.5ex]
\begin{itemize}[parsep=0.5ex]
	\item 个人博客: https://jingbei.li
	\item GitHub: https://github.com/petronny
	\item 语言: 英语 - 熟练(CET-6)
\end{itemize}

%% Reference
%\newpage
%\bibliographystyle{IEEETran}
%\bibliography{mycite}
\end{document}
